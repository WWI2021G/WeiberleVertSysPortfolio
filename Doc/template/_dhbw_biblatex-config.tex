%kein Punkt am Ende bei \footcite
%http://www.golatex.de/footcite-ohne-punkt-am-schluss-t4865.html
\renewcommand{\bibfootnotewrapper}[1]{\bibsentence#1}


%Reihenfolge der Autorennamen
%   
% http://golatex.de/viewtopic,p,80448.html#80448
% Argumente: siehe http://texwelt.de/blog/modifizieren-eines-biblatex-stils/
\DeclareNameFormat{sortname}{% Bibliographie
  \ifnum\value{uniquename}=0 % Normalfall
    \ifuseprefix%
      {%
         \usebibmacro{name:family-given}
           {\namepartfamily}
           {\namepartgiveni}
           {\namepartprefix}
           {\namepartsuffixi}%
       }
      {%
         \usebibmacro{name:family-given}
           {\namepartfamily}
           {\namepartgiveni}
           {\namepartprefixi}
           {\namepartsuffixi}%
       }%
  \fi
  \ifnum\value{uniquename}=1% falls nicht eindeutig, abgek. Vorname 
      {%
         \usebibmacro{name:family-given}
           {\namepartfamily}
           {\namepartgiveni}
           {\namepartprefix}
           {\namepartsuffix}%
       }%
  \fi
  \ifnum\value{uniquename}=2% falls nicht eindeutig, ganzer Vorname 
      {%
         \usebibmacro{name:family-given}
           {\namepartfamily}
           {\namepartgiven}
           {\namepartprefix}
           {\namepartsuffix}%
       }%
  \fi   
  \usebibmacro{name:andothers}}

\DeclareNameFormat{labelname}{% für Zitate
  \ifnum\value{uniquename}=0 % Normalfall
    \ifuseprefix%
      {%
         \usebibmacro{name:family-given}
           {\namepartfamily}
           {\empty}
           {\namepartprefix}
           {\namepartsuffixi}%
       }
      {%
         \usebibmacro{name:family-given}
           {\namepartfamily}
           {\empty}
           {\namepartprefixi}
           {\namepartsuffixi}%
       }%
  \fi
  \ifnum\value{uniquename}=1% falls nicht eindeutig, abgek. Vorname 
      {%
         \usebibmacro{name:family-given}
           {\namepartfamily}
           {\namepartgiveni}
           {\namepartprefix}
           {\namepartsuffix}%
       }%
  \fi
  \ifnum\value{uniquename}=2% falls nicht eindeutig, ganzer Vorname 
      {%
         \usebibmacro{name:family-given}
           {\namepartfamily}
           {\namepartgiven}
           {\namepartprefix}
           {\namepartsuffix}%
       }%
  \fi   
  \usebibmacro{name:andothers}}
      
  
\DeclareFieldFormat{extrayear}{% = the 'a' in 'Jones 1995a'
  \iffieldnums{labelyear}
    {\mknumalph{#1}}
    {\mknumalph{#1}}}        

\renewcommand*{\multinamedelim}{\addslash}
\renewcommand*{\finalnamedelim}{\addslash}
\renewcommand*{\multilistdelim}{\addslash}
\renewcommand*{\finallistdelim}{\addslash}

\renewcommand{\nameyeardelim}{~}

% Literaturverzeichnis: Doppelpunkt zwischen Name (Jahr): Rest 
% http://de.comp.text.tex.narkive.com/Tn1HUIXB/biblatex-authoryear-und-doppelpunkt
\renewcommand{\labelnamepunct}{\addcolon\addspace}

% damit die Darstellung für Vollzitate von Primärquellen in 
% Fußnoten später auf "nicht fett" geändert werden kann 
% (nur für Zitate von Sekundärliteratur relevant)
\newcommand{\textfett}[1]{\textbf{#1}}

% für Zitate von Sekundärliteratur:
\newcommand{\footcitePrimaerSekundaer}[4]{%
  \renewcommand{\textfett}[1]{##1}%
  \footnote{\fullcite[#2]{#1}, zitiert nach \cite[#4]{#3}}%  
  \renewcommand{\textfett}[1]{\textbf{##1}}%
}

% Im Literaturverzeichnis: Autor (Jahr) fett
\renewbibmacro*{author}{%
  \ifboolexpr{%
    test \ifuseauthor%
    and
    not test {\ifnameundef{author}}
  }
    {\usebibmacro{bbx:dashcheck}
       {\bibnamedash}
       {\usebibmacro{bbx:savehash}%
        \textfett{\printnames{author}}%
        \iffieldundef{authortype}
          {\setunit{\addspace}}
          {\setunit{\addcomma\space}}}%
     \iffieldundef{authortype}
       {}
       {\usebibmacro{authorstrg}%
        \setunit{\addspace}}}%
    {\global\undef\bbx@lasthash
     \usebibmacro{labeltitle}%
     \setunit*{\addspace}}%
  \textfett{\usebibmacro{date+extrayear}}}

% Sonderfall: Quelle ohne Autor, aber mit Herausgeber
% Name des Herausgebers wird fett gedruckt
\renewbibmacro*{bbx:editor}[1]{%
  \ifboolexpr{%
    test \ifuseeditor%
    and
    not test {\ifnameundef{editor}}
  }
    {\usebibmacro{bbx:dashcheck}
       {\bibnamedash}
       {\textfett{\printnames{editor}}%
        \setunit{\addcomma\space}%
        \usebibmacro{bbx:savehash}}%
     \usebibmacro{#1}%
     \clearname{editor}%
     \setunit{\addspace}}%
    {\global\undef\bbx@lasthash
     \usebibmacro{labeltitle}%
     \setunit*{\addspace}}%
  \textfett{\usebibmacro{date+extrayear}}}

% Anpassungen für deutsche Sprache
\DefineBibliographyStrings{ngerman}{%
	nodate = {{o.J.}},
	urlseen = {{Abruf:}},
	ibidem = {{ebenda}}
}

% keine Anführungszeichen beim Titel im Literaturverzeichnis
\DeclareFieldFormat[article,book,inbook,inproceedings,manual,misc,phdthesis,thesis,online,report]{title}{#1\isdot}

\newcommand{\literaturverzeichnis}{%
% nur Literaturverzeichnis
% (als eigenes Kapitel)
\phantomsection
\addcontentsline{toc}{chapter}{Literaturverzeichnis}
\spezialkopfzeile{Literaturverzeichnis}
\defbibheading{lit}{\chapter*{Literaturverzeichnis}}
\label{chapter:quellen}
\printbibliography[heading=lit,notkeyword=ausblenden]
}