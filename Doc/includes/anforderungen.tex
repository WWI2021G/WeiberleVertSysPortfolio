\chapter{Anforderungen} % (fold)
\label{chap:Anforderungen}

Im Folgenden sollen die Anforderungen der Portfolioaufgabe von Verteilte Systeme aus dem 4. Semester aufgelistet und erklärt werden.
Außerdem soll gezeigt werden, wo und wie die Anforderungen erfüllt wurden.

\section{Generelle Anforderungen} % (fold)
\label{sec:Generelle Anforderungen}

In diesem Abschnitt sollen alle Anforderungen aufgelistet werden, die für den gesamten Programmentwurf gelten sollen.

Bestellungen sollen \glqq möglichst effizient und fehlerfrei erfasst werden.\grqq 

Bildschirmmasken und erzeugte Ergebnisse sollten \glqq zumindest 'ansprechend' aussehen.\grqq 

\glqq Eine dauerhafte Datenspeicherung (z. B. unter Verwendung von Datenbanken) ist
nicht erforderlich, selbstverständlich müssen aber die eingegebenen Daten zwi-
schengespeichert werden, um die Rechnungserstellung zu ermöglichen.\grqq

\glqq Eine rudimentäre Fehlerprüfung bei der Datenerfassung, die gegebenenfalls offen-
sichtliche Eingabefehler erkennt und verhindert, ist notwendig.\grqq

JSPs und Servlets müssen jeweils mindestens einmal verwendet werden.

In mindestens einem Fall muss die Bildschirmmaske aus einem Java-Programm heraus generiert werden.

Es ist eine Programmbeschreibung notwendig.
Diese muss den Programmaufbau und das Zusammenspiel der einzelnen Komponenten enthalten.
Screenshots zur Erklärung.
Programmcode kommentieren.

Lösung als zip Datei mit Inhalt: war-Datei, und alle JSPs und Servlets.

% section Generelle Anforderungen (end)

\section{Mehrere Tische} % (fold)
\label{sec:Mehrere Tische}

Die Anforderung lautet: \glqq Selbstverständlich gibt es nicht nur einen Tisch im Lokal.\grqq

Um diese Anforderung zu erfüllen erlaubt meine Webseite eine Auswahl von 5 verschiedenen Tischen, die von 1 bis 5 durchnummeriert sind.

% section Mehrere Tische (end)

\section{Bestellungen} % (fold)
\label{sec:Bestellungen}

\subsection{Kleines Getränk- und Speisenangebot} % (fold)
\label{sub:Kleines Getränk- und Speisenangebot}

% subsection Kleines Getränk- und Speisenangebot (end)

\subsection{Nachträgliche Bestellungen} % (fold)
\label{sub:Nachträgliche Bestellungen}

% subsection Nachträgliche Bestellungen (end)

% section Bestellungen (end)

\section{Rechnungen} % (fold)
\label{sec:Rechnungserzeugung}

\subsection{Getrennte Rechnungen} % (fold)
\label{sub:Getrennte Rechnungen}

% subsection Getrennte Rechnungen (end)

\subsection{Gesamtrechnung} % (fold)
\label{sub:Gesamtrechnung}

% subsection Gesamtrechnung (end)

\subsection{Produkte nicht einzeln führen} % (fold)
\label{sub:Produkte nicht einzeln führen}

% subsection Produkte nicht einzeln führen (end)

\subsection{Zahlungsarten} % (fold)
\label{sub:Zahlungsarten}

% subsection Zahlungsarten (end)

\subsection{Kein Trinkgeld} % (fold)
\label{sub:Kein Trinkgeld}

% subsection Kein Trinkgeld (end)

\section{Rabattgutscheine} % (fold)
\label{sec:Rabattgutscheine}

% section Rabattgutscheine (end)

% section Rechnungserzeugung (end)

% chapter Anforderungen (end)
